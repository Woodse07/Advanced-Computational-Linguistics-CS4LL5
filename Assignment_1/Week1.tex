\documentclass[12pt]{report}
\usepackage{amsmath}
\usepackage{graphicx}
\usepackage{hyperref}
\usepackage[utf8]{inputenc}
\usepackage{listings}

\title{CS4LL5
 Advanced Computational Linguistics}
\author{Séamus Woods \\ 15317173}
\date{23/09/2019}

\begin{document}
\maketitle
\newpage

\section{Question 1}
Consider the following equations:
\begin{itemize}
\item[i]
$P(A,B) = P(A) * P(B)$
\item[ii]
$P(A|B) = P(A)$
\end{itemize}
Show that (i) implies (ii), and also that (ii) implies (i).
\newline
\newline
Answer:
\newline
If $P(A,B)$ happening is the same as $P(A) * P(B)$, then A and B are independent. Similarly if $P(A|B)$ is the same as $P(A)$, then A and B are independent.
\begin{center}
(i) implies (ii):
\end{center}
\begin{center}
$P(A,B) = P(A) * P(B)$
\end{center}
\begin{center}
$P(A,B) = P(A|B) * P(B)$
\end{center}
\begin{center}
$P(A) * P(B) = P(A|B) * P(B)$
\end{center}
\begin{center}
$P(A) = P(A|B)$
\end{center}
\begin{center}
$P(A|B) = P(A)$
\end{center}
\begin{center}
\end{center}
\begin{center}
(ii) implies (i):
\end{center}
\begin{center}
$P(A|B) = P(A)$
\end{center}
\begin{center}
$P(A|B) = P(A,B) / P(B)$
\end{center}
\begin{center}
$P(A) = P(A,B) / P(B)$
\end{center}
\begin{center}
$P(A) * P(B) = P(A,B)$
\end{center}
\begin{center}
$P(A,B) = P(A) * P(B)$
\end{center}

\section{Question 2}
(a) Calculate $P(gw|ps)$:
\newline
Sample space of ps is 30, and out of those 30, gw happened 28 times, therefore $P(gw|ps) = 28/30$.
\newline
The sample space of not-ps is irrelevant since we're working with the probability gw happened \underline{given} ps already happened.
\newline
\newline
(b) Calculate $P(ps|gw)$
\newline
Sample space of gw is 168, and out of that ps occurred 28 times. Therefore $P(ps|gw) = 28/168$.
\newline
The sample space of not-gw is irrelevant here because we are only concerned about the chance ps occurred \underline{given} gw has already occurred.

\section{Question 3}
Let vmel stand for Speaker = 'Victor Meldrew'.
\newline
Let dbi stand for DBI = true.
\newline
\newline
Work out which of vmel or not-vmel is likelier, given dbi, supposing the following probabilities:
\newline
(a) 
\newline
$P(vmel) = 0.01$
\newline
$P(dbi|vmel) = 0.95$
\newline
$P(dbi|not vmel) = 0.01$
\newline
\newline
$P(dbi|vmel)P(vmel) = 0.95 * 0.01 = 0.0095$
\newline
$P(dbi|not vmel)P(not vmel) = 0.01 * 1-0.01 = 0.0099$
\newline
Therefore not-vmel is likelier given dbi.
\newline
\newline
(b)
\newline
$P(vmel) = 0.15$
\newline
$P(dbi|vmel) = 0.95$
\newline
$P(dbi|not vmel) = 0.01$
\newline
\newline
$P(dbi|vmel)P(vmel) = 0.95 * 0.15 = 0.1425$
\newline
$P(dbi|not vmel)P(not vmel) = 0.01 * 1-0.15 = 0.0.0085$
\newline
Therefore vmel is likelier given dbi.
\newline
\newline
(c)
\newline
$P(vmel) = 0.01$
\newline
$P(dbi|vmel) = 0.95$
\newline
$P(dbi|not vmel) = 0.001$
\newline
\newline
$P(dbi|vmel)P(vmel) = 0.95 * 0.01 = 0.0095$
\newline
$P(dbi|not vmel)P(not vmel) = 0.001 * 1-0.01 = 0.00099$
\newline
Therefore vmel is likelier given dbi.
\newline
\newline

\section{Question 4}
Find $P(cool+)$ and $P(cool+|noisy+)$ and conclude from this whether or not $cool+$ is independent of $noisy+$
\newline
\newline
$P(cool+) = 170/500 = 17/50$
\newline
$P(cool+|noisy+) = 62/100$
\newline
A condition of independence is $P(A|B) = P(A)$, so if $cool+$ and $noisy+$ are independent they must follow this rule.
\newline
$P(cool+) = P(cool+|noisy+)$
\newline
$17/50 = 62/100$ is not true, and therefore $cool+$ is not independent of $noisy+$.

\section{Question 5}
With reference to table 2, find $P(cool+|open+)$, $P(cool+|open+, noisy+)$.
\newline
$P(cool+|open+)=90/100$
\newline
$P(cool+|open+,noisy+)=54/60$
\newline
Is $cool+$ conditionally independent of $noisy+$ given $open+$?
\newline
Conditional independence = $P(X|Y,Z) = P(X|Z)$.
\newline
Let $X=cool+$, $Y=noisy+$ and $Z=open+$.
\newline
$P(cool+|open+,noisy+) = P(cool+,open+)$
\newline
$90/100 = 54/60$
\newline
$0.9 = 0.9$
\newline
Therefore $cool+$ is conditionally independent of $noisy+$ given $open+$.
\newline
\newline
With reference to table 3, find $P(cool+|open-)$ and $P(cool+|open-,noisy+)$.
\newline
$P(cool+|open-) = 80/400$
\newline
$P(cool+|open-,noisy+) = 8/40$
\newline
Is $cool+$ conditionally independent of $noisy+$ given $open-$?
\newline
$P(cool+|open-,noisy+) = P(cool+|open-)$
\newline
$8/40 = 80/400$
\newline
$0.2 = 0.2$
\newline
Therefore, $cool+$ is conditionally independent of $noisy+$ given $open-$.

\end{document}