\documentclass[12pt]{report}
\usepackage{amsmath}
\usepackage{graphicx}
\usepackage{hyperref}
\usepackage[utf8]{inputenc}
\usepackage{listings}

\title{CS4LL5
 Advanced Computational Linguistics}
\author{Aplying brute force EM on IBM model 1 \\ Séamus Woods \\ 15317173}
\date{11/11/2019}

\begin{document}
\maketitle
\newpage

\section{Iteration 1}
From iteration 1, we have:
\newline
\begin{center}
\begin{tabular}{c c c c}
$tr(o|s)$ & gr & ho & the \\
ca & $\frac{1}{2}$ & $\frac{1}{2}$ & $\frac{1}{2}$ \\\\
ve & $\frac{1}{2}$ & $\frac{1}{4}$ & $0$ \\\\
la & $0$ & $\frac{1}{4}$ & $\frac{1}{2}$ \\\\
\end{tabular}
\end{center}
Our task is to complete two further iterations. 

\section{Iteration 2}
Calculating $num(a^n_n)$...\\\\
$num(a^1_1) = \frac{1}{2} * \frac{1}{4} = \frac{1}{8}$\\\\
$num(a^1_2) = \frac{1}{2} * \frac{1}{2} = \frac{1}{4}$\\\\
$num(a^2_1) = \frac{1}{2} * \frac{1}{2} = \frac{1}{4}$\\\\
$num(a^2_2) = \frac{1}{2} * \frac{1}{4} = \frac{1}{8}$\\\\
From this we can calculate our conditional probabilities $P(a^d_n|o,s)$, we need to sum the $num(a^n)$ by summing across the table and use it as the denominator.\\

$P(a^1_1|o_1,s_1) = \frac{\frac{1}{8}}{(\frac{1}{8} + \frac{1}{4})} = \frac{1}{3}$\\\\

$P(a^1_2|o_1,s_1) = \frac{\frac{1}{4}}{(\frac{1}{8} + \frac{1}{4})} = \frac{2}{3}$\\\\

$P(a^2_1|o_2,s_2) = \frac{\frac{1}{4}}{(\frac{1}{4} + \frac{1}{8})} = \frac{2}{3}$\\\\

$P(a^2_2|o_2,s_2) = \frac{\frac{1}{8}}{(\frac{1}{4} + \frac{1}{8})} = \frac{1}{3}$\\\\

Now we make a count by going through all the alignments and incrementing the count by how many times $o$ is paired with $s$ in the alignment and multiplying that by the above conditional alignment probabilities.\\
We can use the table given to us by the spec, which tells us the counts already. 
\begin{center}
\begin{tabular}{c c c c}
$cnt$ & gr & ho & the \\
ca & $1*\frac{1}{3}$ & $2*\frac{2}{3}$ & $1*\frac{1}{3}$ \\\\
ve & $1*\frac{2}{3}$ & $1*\frac{1}{3}$ & $0$ \\\\
la & $0$ & $1*\frac{1}{3}$ & $1*\frac{2}{3}$ \\\\
\end{tabular}
\end{center}
and for these counts we get $tr(o|s)$ by normalising by column sums:
\begin{center}
\begin{tabular}{c c c c}
$P(o|s)$ & gr & ho & the \\
ca & $\frac{1}{3}$ & $\frac{2}{3}$ & $\frac{1}{3}$ \\\\
ve & $\frac{2}{3}$ & $\frac{1}{6}$ & $0$ \\\\
la & $0$ & $\frac{1}{6}$ & $\frac{2}{3}$ \\\\
\end{tabular}
\end{center}
This is the end of the second iteration, now we'll start the third.

\section{Iteration 3}
Again, we calculate $num(a^n_n)$...\\\\
$num(a^1_1) = \frac{1}{3} * \frac{1}{6} = \frac{1}{18}$\\\\
$num(a^1_2) = \frac{2}{3} * \frac{2}{3} = \frac{4}{9}$\\\\
$num(a^2_1) = \frac{2}{3} * \frac{2}{3} = \frac{4}{9}$\\\\
$num(a^2_2) = \frac{1}{3} * \frac{1}{6} = \frac{1}{18}$\\\\

Again, we calculate conditional probabilities..

$P(a^1_1|o_1,s_1) = \frac{\frac{1}{18}}{(\frac{1}{18} + \frac{4}{9})} = \frac{1}{9}$\\\\

$P(a^1_2|o_1,s_1) = \frac{\frac{4}{9}}{(\frac{1}{18} + \frac{4}{9})} = \frac{8}{9}$\\\\

$P(a^2_1|o_2,s_2) = \frac{\frac{4}{9}}{(\frac{1}{18} + \frac{4}{9})} = \frac{8}{9}$\\\\

$P(a^2_2|o_2,s_2) = \frac{\frac{1}{18}}{(\frac{1}{18} + \frac{4}{9})} = \frac{1}{9}$\\\\

Count multiplied by conditional probabilities...\\
\begin{center}
\begin{tabular}{c c c c}
$cnt$ & gr & ho & the \\
ca & $1*\frac{1}{9}$ & $2*\frac{8}{9}$ & $1*\frac{1}{9}$ \\\\
ve & $1*\frac{8}{9}$ & $1*\frac{1}{9}$ & $0$ \\\\
la & $0$ & $1*\frac{1}{9}$ & $1*\frac{8}{9}$ \\\\
\end{tabular}
\end{center}
and for these counts we get $tr(o|s)$ by normalising by column sums:
\begin{center}
\begin{tabular}{c c c c}
$P(o|s)$ & gr & ho & the \\
ca & $\frac{1}{9}$ & $\frac{8}{9}$ & $\frac{1}{9}$ \\\\
ve & $\frac{8}{9}$ & $\frac{1}{18}$ & $0$ \\\\
la & $0$ & $\frac{1}{18}$ & $\frac{8}{9}$ \\\\
\end{tabular}
\end{center}
This is the end of the third iteration.

\section{Conclusion}
History over the 3 iterations.
\begin{center}
\begin{tabular}{c c c c c c}
Obs & Src & 0 & 1 & 2 & 3 \\
green & casa & $\frac{1}{3}$ & $\frac{1}{2}$ & $\frac{1}{3}$ & $\frac{1}{9}$\\\\
house & casa & $\frac{1}{3}$ & $\frac{1}{2}$ & $\frac{2}{3}$ & $\frac{8}{9}$\\\\
the & casa & $\frac{1}{3}$ & $\frac{1}{2}$ & $\frac{1}{3}$ & $\frac{1}{9}$\\\\
green & verde & $\frac{1}{3}$ & $\frac{1}{2}$ & $\frac{2}{3}$ & $\frac{8}{9}$\\\\
house & verde & $\frac{1}{3}$ & $\frac{1}{4}$ & $\frac{1}{6}$ & $\frac{1}{18}$\\\\
the & verde & $\frac{1}{3}$ & 0 & 0 & 0\\\\
green & la & $\frac{1}{3}$ & 0 & 0 & 0\\\\
house & la & $\frac{1}{3}$ & $\frac{1}{4}$ & $\frac{1}{6}$ & $\frac{1}{18}$\\\\
the & la & $\frac{1}{3}$ & $\frac{1}{2}$ & $\frac{2}{3}$ & $\frac{8}{9}$\\\\
\end{tabular}
\end{center}

Hopefully it's clear from the table that the $tr(o|s)$ should converge to:
\begin{center}
\begin{tabular}{c c c c}
$P(o|s)$ & gr & ho & the \\
ca & $0$ & $1$ & $0$ \\\\
ve & $1$ & $0$ & $0$ \\\\
la & $0$ & $0$ & $1$ \\\\
\end{tabular}
\end{center}

\end{document}